\documentclass[a4paper,11pt]{article}

\usepackage{fullpage}
\usepackage[utf8]{inputenc}
\usepackage[T1]{fontenc}
\usepackage{amsmath}
\usepackage{graphicx}
\usepackage{booktabs}
\usepackage{array}
\usepackage{xcolor}
\usepackage{hyperref}

\hypersetup{colorlinks=true, linkcolor=blue!60!black, urlcolor=blue!60!black}

\title{\textbf{Projet CryptoShell --- Rapport d'analyse}\\
\large TP Not\'e S\'ecurit\'e Informatique --- L3 Informatique}
\author{Nom : \underline{\hspace{5cm}} \quad Pr\'enom : \underline{\hspace{5cm}} \\[0.5em]
Num\'ero \'etudiant : \underline{\hspace{5cm}}}
\date{Date de rendu : 08/03/2026}

\begin{document}
\maketitle

\noindent\textbf{Instructions :} Remplissez chaque section ci-dessous. Le rapport est \'evalu\'e sur la \textbf{qualit\'e de l'analyse} et la \textbf{pr\'ecision des r\'eponses}, pas sur la longueur. Soyez concis mais rigoureux.

\bigskip

%%%%%%%%%%%%%%%%%%%%%%%%%%%%%%%%%%%%%%%%%%%%%%%%%%
\section*{R1 --- Classification d'Adleman (2 points)}

\textbf{Question :} O\`u se situe CryptoShell dans la classification d'Adleman ? Est-ce un programme simple ou auto-reproducteur ? Justifiez.

\bigskip
\noindent\textit{Votre r\'eponse :}
\vspace{4cm}

%%%%%%%%%%%%%%%%%%%%%%%%%%%%%%%%%%%%%%%%%%%%%%%%%%
\section*{R2 --- Cycle de vie (4 points)}

\textbf{Question :} Dessinez (en ASCII art ou sch\'ema) le cycle de vie complet de CryptoShell en identifiant les 3 phases classiques (infection, incubation, maladie). Indiquez quelle phase du TP correspond \`a chaque \'etape.

\bigskip
\noindent\textit{Votre r\'eponse :}
\vspace{6cm}

%%%%%%%%%%%%%%%%%%%%%%%%%%%%%%%%%%%%%%%%%%%%%%%%%%
\section*{R3 --- Virus, Ver ou Cheval de Troie ? (3 points)}

\textbf{Question :} CryptoShell tel qu'impl\'ement\'e est-il un virus, un ver, ou un cheval de Troie ? Comment le modifier pour qu'il devienne un ver (propagation r\'eseau) ? Comment le transformer en cheval de Troie ?

\bigskip
\noindent\textit{Votre r\'eponse :}
\vspace{5cm}

%%%%%%%%%%%%%%%%%%%%%%%%%%%%%%%%%%%%%%%%%%%%%%%%%%
\section*{R4 --- Comparaison avec les virus de documents (3 points)}

\textbf{Question :} Un macro-virus Word et CryptoShell ciblent tous deux des fichiers de donn\'ees. Quelles sont les similitudes et diff\'erences fondamentales (vecteur d'infection, format cible, interpr\'eteur, persistance) ?

\bigskip
\noindent\textit{Votre r\'eponse :}
\vspace{5cm}

%%%%%%%%%%%%%%%%%%%%%%%%%%%%%%%%%%%%%%%%%%%%%%%%%%
\section*{R5 --- Tableau des IoC par phase (4 points)}

\textbf{Question :} Remplissez le tableau ci-dessous pour chaque phase du projet.

\bigskip
\begin{center}
\begin{tabular}{|c|p{3.5cm}|p{4cm}|p{3.5cm}|}
\hline
\textbf{Phase} & \textbf{Technique utilis\'ee} & \textbf{IoC d\'etectable} & \textbf{Type de d\'etection} \\
\hline
1 --- Reconnaissance & & & \\[1.5em]
\hline
2 --- Chiffrement XOR & & & \\[1.5em]
\hline
3 --- Propagation & & & \\[1.5em]
\hline
4 --- D\'eclencheur & & & \\[1.5em]
\hline
5 --- Furtivit\'e & & & \\[1.5em]
\hline
\end{tabular}
\end{center}

\bigskip
\noindent\textbf{Strat\'egie de d\'etection combin\'ee :} Proposez une strat\'egie qui combine les 3 approches (signature, heuristique, comportementale).

\bigskip
\noindent\textit{Votre r\'eponse :}
\vspace{4cm}

%%%%%%%%%%%%%%%%%%%%%%%%%%%%%%%%%%%%%%%%%%%%%%%%%%
\section*{R6 --- R\'eflexion \'ethique (4 points)}

\textbf{Question :} En quoi l'\'etude de la construction de malware dans un cadre acad\'emique est-elle b\'en\'efique pour la cybers\'ecurit\'e ? Citez au moins un exemple concret o\`u la compr\'ehension des techniques offensives a permis d'am\'eliorer les d\'efenses.

\bigskip
\noindent\textit{Votre r\'eponse :}
\vspace{5cm}

\end{document}
